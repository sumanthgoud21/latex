\documentclass[12pt,letter paper]{article}
\usepackage{siunitx}
\usepackage{setspace}
\usepackage{gensymb}
\usepackage{xcolor}
\usepackage{caption}
%\usepackage{subcaption}
\doublespacing
\singlespacing
\usepackage[none]{hyphenat}
\usepackage{amssymb}
\usepackage{relsize}
\usepackage[cmex10]{amsmath}
\usepackage{mathtools}
\usepackage{amsmath}
\usepackage{commath}
\usepackage{amsthm}
\interdisplaylinepenalty=2500
%\savesymbol{iint}
\usepackage{txfonts}
%\restoresymbol{TXF}{iint}
\usepackage{wasysym}
\usepackage{amsthm}
\usepackage{mathrsfs}
\usepackage{txfonts}
\let\vec\mathbf{}
\usepackage{stfloats}
\usepackage{float}
\usepackage{cite}
\usepackage{cases}
\usepackage{subfig}
%\usepackage{xtab}
\usepackage{longtable}
\usepackage{multirow}
%\usepackage{algorithm}
\usepackage{amssymb}
%\usepackage{algpseudocode}
\usepackage{enumitem}
\usepackage{mathtools}
%\usepackage{eenrc}
%\usepackage[framemethod=tikz]{mdframed}
\usepackage{listings}
%\usepackage{listings}
\usepackage[latin1]{inputenc}
%%\usepackage{color}{   
%%\usepackage{lscape}
\usepackage{textcomp}
\usepackage{titling}
\usepackage{hyperref}
%\usepackage{fulbigskip}   
\usepackage{tikz}
\usepackage{graphicx}
\lstset{
  frame=single,
  breaklines=true
}
\let\vec\mathbf{}
\usepackage{enumitem}
\usepackage{graphicx}
\usepackage{siunitx}
\let\vec\mathbf{}
\usepackage{enumitem}
\usepackage{graphicx}
\usepackage{enumitem}
\usepackage{tfrupee}
\usepackage{amsmath}
\usepackage{amssymb}
\usepackage{mwe} % for blindtext and example-image-a in example
\usepackage{wrapfig}
\graphicspath{{figs/}}
\providecommand{\mydet}[1]{\ensuremath{\begin{v\matrix}#1\end{v\matrix}}}
\providecommand{\myvec}[1]{\ensuremath{\begin{b\matrix}#1\end{b\matrix}}}
\providecommand{\cbrak}[1]{\ensuremath{\left\{#1\right\}}}
\providecommand{\brak}[1]{\ensuremath{\left(#1\right)}}
\providecommand{\sbrak}[1]{\ensuremath{{}\left[#1\right]}}
\title{Questions}
\date{}
\author{}
\begin{document}
\maketitle{}
\begin{center}
	\section*{Geometry}
\end{center}
\begin{enumerate}
    \item Let $ABCD$ be a convex quadrilateral with perpendicular diagonals. If $AB = 20$, $BC = 70$, and $CD = 90$, then what is the value of $DA$?
    \item In a triangle with integer side lengths, one side is three times as long as a second side, and the length of the third side is $17$. What is the greatest possible perimeter of the triangle?
    \item In a triangle $ABC$, $X$ and $Y$ are points on the segments $AB$ and $AC$, respectively, such that $ AX : XB = 1 : 2 $ and $ AY : YC = 2 : 1.$If the area of triangle $AXY$ is $10$, then what is the area of triangle $ABC$?
    \item Let $ABCD$ be a convex quadrilateral with $\angle DAB = \angle BDC = 90^\degree$. Let the incircles of triangles $ABD$ and $BCD$ touch $BD$ at $P$ and $Q$, respectively, with $P$ lying in between $B$ and $Q$. If $AD = 999$ and $PQ = 200$, then what is the sum of the radii of the incircles of triangles $ABD$ and $BDC$?
    \item Let $XOY$ be a triangle with $\angle XOY = 90^\degree$. Let $M$ and $N$ be the midpoints of legs $OX$ and $OY$, respectively. Suppose that $XN = 19$ and $YM = 22$. What is $XY$?
\end{enumerate}
\begin{center}
	\section*{Number System}
\end{center}
\begin{enumerate}
   \item A natural number $k$ is such that $k^2 \textless 2014 \textless (k+1)^2$. What is the largest prime factor of $k$?
   \item The first term of a sequence is $2014$. Each succeeding term is the sum of the cubes of the digits of the previous term. What is the $2014^{th}$ term of the sequence?
   \item What is the smallest possible natural number $n$ for which the equation $x^2 - nx + 2014 = 0$ has integer roots?
   \item If $x^{\brak{x^4}} = 4$, what is the value of $x^{\brak{x^2}} + x^{\brak{x^8}}$?
   \item Let $S$ be a set of real numbers with mean $M$. If the means of the sets $S \cup \{15\}$ and $S \cup \{15, 1\}$ are $M + 2$ and $M + 1$, respectively, then how many elements does $S$ have?
   \item Natural numbers $k, l, p,$ and $q$ are such that $a$ and $b$ are roots of the equation $x^2 - kx + l = 0$ such that $a + \frac{1}{b}$ and $b + \frac{1}{a}.$What is the sum of all possible values of $q$?
   \item For natural numbers $x$ and $y$, let $\brak{x, y}$ denote the greatest common divisor of $x$ and $y$. How many pairs of natural numbers $x$ and $y$ with $x \leq y$ satisfy the equation $xy = x + y + \brak{x, y}$?
   \item For how many natural numbers $n$ between $1$ and $2014$ \brak{both inclusive} is $\frac{8n}{9999 - n}$ an integer?
   \item For a natural number $b$, let $N\brak{b}$ denote the number of natural numbers $a$ for which the equation $x^2 + ax + b = 0$ has integer roots. What is the smallest value of $b$ for which $N\brak{b} = 20$?
    \item One morning, each member of Manjul's family drank an 8-ounce mixture of coffee and milk. The amounts of coffee and milk varied from cup to cup, but were never zero. Manjul drank $\frac{1}{7}$-th of the total amount of milk and $\frac{2}{17}$-th of the total amount of coffee. How many people are there in Manjul's family?
\end{enumerate}
\begin{center}
	\section*{Algebraic Equations}
\end{center}
\begin{enumerate}
\item If real numbers $a, b, c, d, e$ satisfy
    \[
        a + 1 = b + 2 = c + 3 = d + 4 = e + 5 = a + b + c + d + e + 3,
    \]
    what is the value of $a^2 + b^2 + c^2 + d^2 + e^2$?
    \item Let $x_1, x_2, \cdots, x_{2014}$ be real numbers different from $1$, such that$x_1 + x_2 + \cdots + x_{2014} = 1$ and
    \[
        \frac{x_1}{1 - x_1} + \frac{x_2}{1 - x_2} + \cdots + \frac{x_{2014}}{1 - x_{2014}} = 1.
    \]
    What is the value of
    \[
        \frac{x_1^2}{1 - x_1} + \frac{x_2^2}{1 - x_2} + \frac{x_3^2}{1 - x_3} + \cdots + \frac{x_{2014}^2}{1 - x_{2014}}?
    \]
\end{enumerate}
\begin{center}
	\section*{Discrete}
\end{center}
\begin{enumerate}
	\item What is the number of ordered pairs $\brak{A, B}$ where $A$ and $B$ are subsets of $\{1, 2, \ldots, 5\}$ such that neither $A \subseteq B$ nor $B \subseteq A$?
\end{enumerate}
\begin{center}
	\section*{Functions}
\end{center}
\begin{enumerate}
	\item Let $f$ be a one-to-one function from the set of natural numbers to itself such that $f\brak{mn} = f\brak{m} f\brak{n}$ for all natural numbers $m$ and $n$. What is the least possible value of $f\brak{999}$?
\end{enumerate}
\begin{center}
	\section*{Trignometry}
\end{center}
\begin{enumerate}
\item In a triangle $ABC$, let $I$ denote the incenter. Let the lines $AI$, $BI$, and $CI$ intersect the incircle at $P$, $Q$, and $R$, respectively. If $\angle BAC = 40^\degree$, what is the value of $\angle QPR$ in degrees?
\end{enumerate}
\end{document}
